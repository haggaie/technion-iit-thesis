%
% You may wish to use some of the following options of the iitthesis
% package:
%
% fullpageDraft      avoid the margins necessary for proper binding and
%   just view or print a draft.
% beforeDefense      make the personal acknowledgements invisible;
%   use this to print the copies you submit initially to the grad school
%   for sending to the opponent panel, i.e. thesis readers (who shouldn't
%   see those parts). For the final submission, after having successfully
%   defended - drop this option.
% noabbrevs          avoid generation of a notation & abbreviations list
%
% Additionally, you must specify the degree for which you're writing
% your thesis (MSc/PhD/MArch etc.)
%
\documentclass[PhD,fullpageDraft]{misc/iitthesis}


% Definitions of info fields for the thesis - subject, advisor,
% faculty, acknowledgements, etc. etc. The thesis-fields file 
% contains Hebrew text, and should use the UTF-8 character set
% encoding (not iso-8859-8-i or windows codepage 1255).
\include{misc/thesis-fields}

% Personal acknowledgements (are only used for the post-exam
% version)
\include{front/personal-acks}

% A separate file for the abstract - in English and in Hebrew, so
% you must make sure it's also in the UTF-8 character set encoding.
%
\include{front/abstract}

% Comment this if you do not want a list of abbreviations and acronyms
% (if you have used the noabbrevs option).
% Use this file to create "glossary entries" for abbreviations and acronyms.
% The entries defined here don't necessarily have to be used in the thesis.

% For this file to compile (and the example text in the main/prelims.tex file),
% the package glossaries-extra is required. It is automatically included unless
% the noabbrevs class option is used.

% Print long form of acronym first with short in parenthesis
\setabbreviationstyle[acronym]{long-short}

\newacronym[%
    description=``The Senate and People of Rome'']%
    {spqr}{SPQR}{Senātus Populusque Rōmānus}

\newacronym[%
    description=``A technology used in data storage devices'']%
    {smart}{SMART}{Self-Monitoring, Analysis and Reporting Technology}

\newabbreviation[%
  description=]
  {tla}% the key of the acronym (used in \gls macro for example)
  {TLA}% the short form of the acronym
  {three-letter acronym}% the long form of the acronym

\newacronym[%
  description=a four-letter acronym]
  {etla}% the key of the acronym (used in \gls macro for example)
  {ETLA}% the short form of the acronym
  {extended three-letter acronym}% the long form of the acronym

\newglossaryentry{symb:c}{%
  name=$c$,%
  description=the speed of light%
}

\newglossaryentry{symb:a-b}{%
  name=\ensuremath{a \pm b},%
  description=the closed interval \ensuremath{\left[a-b,a+b\right]}%
}

\newglossaryentry{supercali}{%
  name=supercalifragilisticexpialidocious,
  description=%
    Atoning for being educable through delicate beauty.
    Something to say when you have nothing to say.}

% --------------------------------

% Commands below will control the behavior/appearance of the list of abbreviations and acronyms

% Uncomment this command to have _all_ abbreviations and acronyms defined
% in this file appear in the final list - rather than just the ones you
% use in the thesis
%\keepUnusedAbbreviations


% Additional machinery relevant to any thesis
% (it's not part of the document class because it's not absolutely
% necessary and not everyone might like it)
\usepackage{misc/iitthesis-extra}

% Definitions useful for anything you write, which you also
% include in any articles, presentations, HW assignments and other
% documents. May contains macros for notation algebra, logic,
% calculus and other fields, as well as general shorthands and
% LaTeX tricks, and package use commands
\include{misc/my-general}

% Definitions, settings and tweaks for this thesis specifically
\include{misc/my-thesis-specific}

% If you are using WinEdt, and using a publication list on the the
% acknowledgements page, and are having problems getting your document
% to compile with the 'PDFLaTeXify' button, try uncommenting the
% following two lines;
% Also, you will need to PDFLaTeXify at least twice, as WinEdt misses
% an extra run. See also:
% http://tex.stackexchange.com/q/41727/5640
\usepackage{multibib}
\newcites{pubinfo}{Acknowledgement page references}
\def\iitthesisextramultibibdefs{}

\begin{document}

% Front Matter
% ------------

% The following command will typeset the outer cover page, the
% inner title page, the acknowledgements page, etc. - everything
% up to but not including the introduction
\makefrontmatter

% Main Matter
% ------------
%
% To conform to Technion regulations, the main matter should begin
% with an introduction (including a survey of relevant past work):
%
\include{main/intro}

%
% and then cover:
% - The methods used in the research
% - The research results
% - Discussion and conclusions from the results
%
% but not necessarily with a specific chapter for each of them.
%
% Then you have your main chapters (although these might still
% include an initial chapter on technical preliminaries, experimental
% system setup, and/or a final chapter with summary, discussion and further
% research direction or questions)

\chapter{Preliminaries}
\label{chap:prelims}

A preliminaries chapter is not necessary, but it may be a good idea to use it for presenting your theoretical/mathematical framework in a more detailed and technical way than the introduction, and to perhaps establish some basic lemmata/observations common to multiple chapters of your thesis.

\section{Some section}

Let's define some concept we'll be using throughout the thesis.

\begin{definition}
The \emph{von Neumann model} of a computer, also known as the \emph{Princeton architecture} is an architecture for digital computers, which consists of a processing units, containing an ALU and processing registers; a control unit consisting of an instruction register and a program counter; a memory unit which stores both data and instructions; and input-and-output mechanisms.
\end{definition}

\section{Acronyms and abbreviations}

Your thesis will typically have a set of significant terms, abbreviations and acronyms. Technion guidelines mandate that you place a list of these at the beginning of your thesis; and that they be defined upon first use. And, indeed, if you followed read this sample thesis carefuly thus far you should have seen ``\nameref{chap:notation-and-abbreviations}'' following the abstract.

When writing your thesis, collect such terms and their definitions in the \texttt{front/abbrevs.tex} file --- using the commands \verb|\newacronym|,  \verb|\newabbreviation| and \verb|\newglossaryentry|; the latter command is used for symbols and short, but unabbreviated, terms.

In the body of your thesis, your first use of a term will typically be where you want to also include the text of its definition. You don't need to repeat the definition you've already entered! Let's explain with an example: You've defined the term \gls{DIY} beforehand; when using it, you invoke the command \verb|\gls{DIY}|.\footnote{The 
\texttt{\textbackslash{}gls} command originates in the \texttt{glossaries-extra} package, which is used to automate the handling of notation \& abbreviations.} This command does several things:
\begin{itemize}
	\item It ensures the term \gls{DIY} is included in the list of Notation \& Abbreviation, at the beginning of the thesis; the entry for the term will also include the page on which it first appears;
	\item It produces the definition text; and finally
	\item It adds the defined term --- \gls{DIY} --- in parentheses, after the definition.
\end{itemize}
In later invocations of \verb|\gls{DIY}|, only the short form (\gls{DIY}) will be printed, not the definition, and no parentheses. (This also means that if you move text around in your thesis you don't have to worry about defining on first use - that's already taken care of.) 

\chapter{A main chapter}
\label{chap:firstchap}

\section{Introduction}

You might have a per-chapter mini-intro, possibly tying in to the relevant part of the general intro.

\section{A section}

\lipsum[1]

Let's cite a source: \cite{Yao1977}. And now,  let's introduce a (numbered) equation...
\begin{align}
\label{eq:emc2}
e &= mc^2
\end{align}

\begin{note}
	Are you seeing a problem with the equation numbering? In some TeX processors and on some platforms, there may be a layout error, so that instead of ``(3.1)'' you get ``)3.1)'' or some other flipping of directions. Specifically, \url{http://overleaf.com} suffers from this problem (at least until 2020). Please make sure and use an appropriate TeX distribution that's up-to-date. Specifically, recent TeXLive versions work fine. On Overleaf, you can switch TeXLive versions using the main menu.
\end{note}

In \autoref{sec:thm-like} below, we will state some theorems.

\section{Results... and theorem-like environments}
\label{sec:thm-like}

What's so special about the theorem-like environments used here? There are several packages which offer the capability of defining these, mainly \texttt{amsthm}, \texttt{ntheorem} and also \texttt{thmtools}. (The last is probably also the most feature-full and versatile, but I'm not familiar with it and the first two are the popular ones.) Many people writing a Technion thesis start with \texttt{amsthm}, only to find out it has conflicts with Hebrew... also, there's the issue of aliasing (same counter for lemmata and theorems, but having \texttt{{\textbackslash}autoref} and similar commands know what they're referencing.) This is all neatly resolved in \texttt{iitthesis-extra.sty} with \texttt{amsthm}-like-looking environments actually done with nthrerom.

\begin{theorem}
\label{thm:first}
This is the first numbered theorem in this thesis.
\end{theorem}

And we can refer to it using \texttt{ref}: \ref{thm:first} and get the number, or use hypertex's \texttt{autoref}: \autoref{thm:first}.

\begin{corollary}
\label{cor:first}
There are no lemmata appearing before theorems in this thesis.
\end{corollary}

\begin{theorem*}
This is the second theorem, unnumbered.
\end{theorem*}

\begin{theorem*}[\protect{\cite[Theorem 2]{Knuth1973}}]
This is an unnumbered theorem cited from elsewhere. \qbfox{1} ... and it was Knuth's dog.
\end{theorem*}

\begin{note}
This is a note environment.  \qbfox{2}
\end{note}

\begin{definition}
\label{def:first}
An \emph{quick brown fox} is a fox which is not only fast and agile but is also characterized by brown fur. Such foxes sometimes tend to jump over lazy dogs.
\end{definition}

\begin{lemma}
\label{lem:first}
This is a lemma. \qbfox{2}
\end{lemma}

Even though \autoref{lem:first} and \autoref{def:first} share the ``same'' counter, when referring to them, their names are used automagically.

Here's a proof of the lemma:
\begin{proof}%[lem:natural:blowups-preserve-distance-on-average]
\lipsum[2]
	It's common to conclude proofs with a QED symbol --- typically a full or an empty black square. To do so, append the command \verb|\qed| after the last sentence of your proof; or, alternatively, you can use some \LaTeX{} trickery in the definition of the proof environment to ensure this symbol is appended to all proofs. This is done in the \texttt{misc/iitthesis-extra.sty} style file, which is used by this template. You will see the result as a black square at the end of this proof environment.
\end{proof}

Now here's a proof of \autoref{thm:first} using the \verb|proofof| environment.
\begin{proofof}[thm:first]
\lipsum[3]
\end{proofof}

\begin{note}There may currently be a problem getting the QED symbol (\verb|\qed|) to appear if your proof environment ends with certain display-mode-math environments, such as \verb|align*|.
\end{note}

\begin{proposition}
\label{prop:first}
This is a proposition environment. \qbfox{2}
\end{proposition}

\begin{observation}
\label{obs:first}
The moon revolves around the earth.
\end{observation}

There are several other theorem-like environments, of various kinds, defined in \texttt{misc/iitthesis-extra.sty}.

\subsection{A subsection}

We've started a subsection. Here is a reference to another chapter: \autoref{chap:prelims} --- realized with the \verb|\autoref| command. If you've used \texttt{iitthesis-extra.sty}, it should ensure the environment name produced by \verb|\autoref| is capitalized (``Chapter'' rather than ``chapter'').

\begin{algorithm}
\caption{A nice algorithm}
\label{alg:first}
\begin{algorithmic}[1]
\FOR{$n$ times}
  \STATE{Do something.}
  \STATE{Do something else.}
\ENDFOR
\STATE{And do one last thing.}
\end{algorithmic}
\end{algorithm}

It is recommended to use \texttt{algorithmicx} over \texttt{algorithmic} for algorithms, like in \autoref{alg:first}, as it has less conflicts with Hebrew babel (regardless of whether you have Hebrew in your algorithms or not). Also, \texttt{iitthesis-extra.sty} provides it with a necessary workaround.

\subsection{A second subsection}

In this subsection we'll have a figure. Remember that the {\LaTeX} compiler can place figures a little before or after where they are defined, according to the placement option choice and depending on the flow of the rest of the text.

\begin{figure}[htb]
  \centering
  \ifpdf
    \includegraphics{graphics/mygraphic1.pdf}
  \else
    \includegraphics{graphics/mygraphic1-for-ps.eps}
  \fi
  \caption{Two circles and a wavy line.}
\end{figure}


\include{main/conclusion}
%
% Add any appendices here; they must come _before_ the bibliography
%
\appendix
%\noappendicestocpagenum
%\addappheadtotoc
\include{main/appendix1}

% Back Matter
% ------------

% The following command will typeset the bibliography,
% then typeset the Hebrew part of the thesis:
% - Cover page
% - Title page
% - Acknowledgements page
%  (NO table of contents or list of figures in Hebrew)
% - (Extended) abstract (1000-2000 words)
%
% based on information you've provided in the thesis-fields file
% (including the relative paths to your bib files). The Hebrew
% content will be typeset in _reverse_page_order_, i.e. first
% in the file will be the last page of the abstract, and the
% Hebrew cover page will be the last page of the file.
%
\makebackmatter

% The resulting PDF can be printed and taken straight to binding,
% i.e. you do not need to flip any pages anywhere. Of course,
% mind the LaTeX error and warning messages, overfull hboxes etc.

\end{document}

